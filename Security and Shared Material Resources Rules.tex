\documentclass{article}
\usepackage[utf8]{inputenc}

\title{Security and Shared Material Resources Rules}
\date{}

\begin{document}

\maketitle

\section*{General Security Rules}
\begin{itemize}
    \item Always wear gloves, a lab coat and protective eyewear, as well as long pants, socks and closed sturdy shoes. On the contrary, avoid gloves and lab coats outside the lab (unless warranted by an exceptional situation for your own safety), take them off to avoid touching doors also used without protective gear.
    \item Wash your hands as you leave the laboratory.
    \item Do not bring in any food or beverage inside the laboratory.
    \item Never store products in the drawers, fume hoods or on the workbenches. Check the designated storage space by CAS number in the Quartzy inventory.
    \item Temporary storage in appropriate bottles of solvents or samples is only allowed with proper labelling following the format ``YourInitials-AAMMJJ'' and the chemical product name+CAS number or sample reference to your lab book number+page. This procedure will be upgraded in the near future to implement the new SIMDUT standards.
\end{itemize}


\section*{Fume hood:}
\begin{itemize}
    \item Always keep the back as clear as possible for proper airflow evacuation.
    \item Glassware that held volatile compounds should be left in the fume hood at least overnight after proper rinsing and/or cleaning, even if it ultimately goes into a trash can.
\end{itemize}


\section*{Shared Ressources:}
\begin{itemize}
    \item We need to share the work space, thus wash the workbenches and the fume hood after using them, every day.
    \item You always need to ask permission before borrowing glassware and fume hoods belonging to another group (gr. Allen and gr. Boudreau).
    \item You should never put back a defective apparatus or articles, it’s really annoying; you need to notify the rest of the group by email and get it repaired or replaced.
\end{itemize}


\section*{Inventory Management of the Chemical Products:}
\begin{itemize}
    \item When you receive a new product, you need to add it to the inventory.  You need to write down its reception date, apply the "CA" sticker and the location where it is stored on the product’s label.
    \item When you run out of product, you need to:
    \begin{itemize}
        \item Prepare a new order and add a note of emptiness in the Quartzy inventory.
    \end{itemize}
\end{itemize}

\section*{Waste}
\begin{itemize}
    \item Regarding the used solvants’ recycling bins, you need to:
    \item Learn the proper procedures for waste disposal. Note that disposal costs of the red trash can for solid toxic waste are high, so it should not be used for regular waste, but safety first. Specifically, only the gloves used to handle cadmium-based nanosemiconductors or the Cd precursors, are to be thrown into the red trash can.
    \item Bring the waste containers under the fume hood to pour the used liquids.
    \item Always close them after using and clean them when needed.
    \item Do not overfill the containers, watch the subtle line for the maximum.
    \item Engage the pick up procedure when it is full.
    \item You need to make sure you deactivate for 24h the Piranha solutions before putting it in the trash bin.  Also, it is preferable to leave the container in the fume hood with the little cap open to avoid suppression risks.
    \item Use the characterization apparatus (POP-0314).
    \item Use the reservation boards provided for this purpose.
    \item Respect the other group’s members’ reservations.
    \item Fill in the user notebook scrupulously.
\end{itemize}



\end{document}
